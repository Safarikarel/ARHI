\section{SUMMARY AND OUTLOOK}
\label{summary}
This article attempts to overview the studies of hard-probes observables exploiting data from the first two runs with Pb--Pb collisions at the CERN LHC, running at the collision energy $\rootsNN = 2.76$~TeV per a nucleon pair. Each of the three major LHC experiments (ALICE, ATLAS and CMS) sampled an integrated luminosity in excess of 100~$\mu$b$^{-1}$, corresponding to about a billion of minimum-bias Pb--Pb collisions. The suppression of high-\pT\ particles, relative to what is measured in proton--proton collisions, is expressed with the \pt\ dependence of the nuclear modification factor \Raa. This reveals a strong suppression at intermediate \pT\ (6--8~GeV), but then a lesser effect, and even a plateau for \pt\ above 40~GeV, with a \Raa\ value of about 0.5--0.6, very similar to that found for jet suppression. The study of identified-particle suppression shows a presence of three \pt\ regions: bulk ($\pt < 2$--$3$~GeV), intermediate (up to 7--8~GeV), and fragmentation (above that); and demonstrates that probably there is no any particle-identified life above \pt\ of 10~GeV. Two particle correlations, despite not being a full jet measurement, provide an insight into the suppression in the away-side region and an enhancement on the near-side,
suggestive of modified in-medium fragmentation.

Heavy flavor, copiously produced at the LHC, is expected to show different energy loss than that of
light quarks and gluons. Early results show a strong suppression of D mesons, comparable with that for light-flavor hadrons. The B-meson production, measured with J$/\Psi$ from B-meson decays, indicates less suppression for beauty compared to charm quarks. Thus, for the first time, the predicted mass-quenching hierarchy is respected. The J$/\psi$ yield is found to be suppressed over a wide rapidity range, although demonstrating qualitatively different centrality behavior compared to RHIC results. The \Jpsi\ suppression at low $\pT$ is largely less than that measured at RHIC, suggesting clearly a new production mechanisms. Models, invoking regeneration processes due to the large charm-quark densities at the LHC, are likely to explain this phenomena. Both, D mesons and \Jpsi, show an azimuthal asymmetry demonstrated with a non-zero $v_2$, which is an indication for charm-quark thermalization. The $\Upsilon$ family confirms the predicted behavior in heavy-ion collisions, with all states being suppressed relative to pp collisions, and the more weakly bound higher states showing stronger suppression than the more tightly bound \PgUa\ state.

The particle production involving hard processes is well calibrated using measurements of electro-weak bosons, such as W and Z, as well as direct photons. All of these particles are found to have cross sections consistent with perturbative QCD calculations scaled by the number of binary collisions, $N_{\rm coll}$. In this context, the modifications of jet yields in (semi-)central heavy-ion collisions can clearly be attributed to the energy loss of partons traversing the hot, dense medium. The jet energy loss has been addressed using several techniques, from dijet $E_{\rm T}$ imbalance to single-jet suppression, both inclusive and differentially in $\varphi$. Jet fragmentation
in heavy-ion collisions has also been measured and found to be modified, especially at large angles with respect to the jet axis. The jet--direct-$\gamma$ correlations shows a similar energy loss effect.

The results presented here are mainly those already published, and many more results are in preparation. Therefore, this review should be seen as a snapshot of the state of the field. The wealth of results from the first two heavy-ion runs, and the upcoming running at higher energy ($\energy = 5.1$ TeV) and luminosity exceeding the design value of $10^{27}$~cm$^{-2}$s$^{-1}$, should provide even deeper insight into the nature and properties of the hot and dense matter formed in nuclear collisions at the LHC.

The progress towards a detailed characterization of this strongly-interacting medium will focus more on hard probes,
studying their interactions with this medium, and on hadronization. These will include measurements of heavy-flavor particles, quarkonium states, real and virtual photons, jets and their correlations with other probes (in particular electro-weak bosons). The production rate of all these processes are significantly larger at the LHC than at previous accelerators. In addition, the interactions with the medium of hard probes is better controlled theoretically than the propagation of soft light partons.

In order to achieve these goals, high precision measurements together with high integrated luminosities
are required. This will give an access to the rare physics channels needed to understand the dynamics of this condensed phase of QCD matter. Therefore, the LHC collaborations are upgrading the current detectors to enhance their vertexing capabilities, and to allow for data taking at substantially higher rates.  The upgrade strategy for
heavy-ion running is formulated under the assumption that, after the second long shutdown in 2018--19, the LHC will progressively increase its luminosity also with Pb beams, eventually reaching an interaction rate of about 50~kHz, corresponding to the luminosity $6 \times 10^{27}$~cm$^{-2}$s$^{-1}$. The proposed plan~\cite{ALICEUpgradeLoI}
envisage to accumulate 10~nb$^{-1}$ of Pb--Pb collisions inspecting ${\cal O}(10^{10})$ interactions, which is necessary to address the proposed physics programme, focused on hard probes, both at low- and high-\pt, as well as on the multi-dimensional analysis of such probes with respect to centrality, event plane, and multi-particle correlations. 