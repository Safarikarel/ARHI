\begin{abstract}
The first results on hard-probe observables in heavy-ion collisions obtained with the CERN Large Hadron Collider (LHC), accelerating lead-ion beams since late 2010, are presented. During the first three years of operation the ALICE, ATLAS and CMS experiments each collected \PbPb\ interaction data corresponding to an integrated luminosity above 100~\mubinv\ at a center-of-mass energy $\rootsNN = 2.76$\TeV\ per nucleon pair, thus exceeding the previously studied collision energies by more than one order of magnitude. With extensive measurements of newly accessible hard probes of the hot and dense medium, fresh insights into the properties of QCD matter under extreme conditions are emerging. A comprehensive overview of the results obtained in heavy-ion collisions at the LHC so far is provided. While the basic picture established at RHIC --- that of a hot, dense medium flowing with a viscosity-to-entropy ratio close to the conjectured lower bound, and which quenches the energy of hard probes, such as jets, heavy-flavours, and quarkonia --- is confirmed at the LHC, the higher-energy data indicate new ways for extracting its properties in more details. These include the first observations of a different strength for the suppression of charm and beauty quarks,  reduced low-\pt\ suppression for \Jpsi, and detailed measurements of high-$E_{\rm T}$ jet suppression and jet-fragmentation modifications. The upcoming high-precision measurements of hard probes in heavy-ion collisions will rely on high-luminosity LHC running with heavy ions and detector upgrades currently being planned.
\end{abstract}
