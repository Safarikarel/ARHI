\section{INTRODUCTION}
\label{introduction}

In 2010 and 2011, the Large Hadron Collider~\cite{Evans:2008zzb} at CERN, 
near Geneva, Switzerland, provided collisions of
fully-stripped lead beams at a nucleon-nucleon center-of-mass energy of $\sqrt{s_{\mathrm{NN}}}=2.76$ TeV,
the highest energy ever available in a laboratory setting.  
The goal of this experimental program was to create an ultra-hot, and ultra-dense state 
of matter typically called the ``quark-gluon plasma'' (QGP), where the degrees of freedom are expected to be the 
quarks and gluons which make up the nucleons that themselves comprise the atomic nucleus.
The RHIC experiments reported the observation of the ``strongly interacting'' sQGP in a series of 
White Papers~\cite{Adams:2005dq,Arsene:2004fa,Back:2004je,Adcox:2004mh}.
in 2005. 
One of the primary diagnostics of the formation for the QGP, and a possible means to measure its microscopic
properties, is by means of ``hard probes'' -- high transverse momentum hadrons, heavy flavor hadrons,
electroweak bosons, and hadronic jets -- which are the products of hard scattering processes among the
individual nucleons.  
The goal of this review is to present the current published data on each of these channels, and assess
what has been learned about the properties of the QGP.
A review of the first heavy ion data from the LHC can be found in a 
previous edition of Annual Reviews~\cite{Muller:2012zq}.

Hard scattering processes have been put to good use in subatomic physics since the
very first experiments by Rutherford and his colleagues scattering alpha particles off of gold foil~\cite{Rutherford}.
Having the nuclear charge distributed over the full atomic volume, as predicted in the so-called
``plum pudding'' model, would lead to small deflections.  Instead, what was seen was a rate of
backwards scattered alpha particles much larger than expected, reflecting the elastic scattering off
of a concentration of the nuclear charge in a small volume.  
Without a probe of sufficiently high energy to probe such small wavelengths, this discovery would 
have been impossible.  

Since then, particle beams of higher and higher energies have been used
to explore the structure of subatomic particles, in particular nuclei and hadrons.
The high energies provided by the electron beam at SLAC in the 1960's led to the
landmark deep inelastic scattering (DIS) experiments on the proton~\cite{Breidenbach:1969kd}.
These measurements were the first to characterize the momentum transfer of the scattering
process using the kinematic variables $Q^2$ and Bjorken $x$.  The virtuality $Q^2$ reflects
the momentum transfer from the electron to the hadron via a virtual photon, and is measured
by the outgoing scattering angle and energy of the final-state electron.
Via the uncertainty principle, $Q^2$ controls the spatial scales resolved by the virtual photon.
The Bjorken $x$ variable is defined as the ratio of squared momentum transfer to the 
invariant four dot-product of the 4-momenta of the virtual photon and incoming proton.
Under the assumption that the nucleon is composed of pointlike charged particles (``partons''),
$x$ was interpreted by Bjorken as the fraction of the proton momentum carried by the 
struct object, and it was expected that cross sections would not vary with $Q^2$~\cite{PhysRev.179.1547}.
The observation of this phenomenon led to the wide acceptance of
%led to the awarding of the 1990 Nobel prize in physics 
%to the experimental team for the discovery of the charged constitutents in the nucleon,
%identified with the 
charged ``quarks'' predicted on the basis of regularities in the observed hadron spectrum.
This solidified the importance of hard scattering as a tool to elucidate hadron structure.

By the 1990's the HERA collider had been built at DESY in Hamburg, Germany, to perform 
searches for new physics coupling to leptons and hadrons, as well as precision DIS 
at previously unavailable beam energies, allowing experimental access to far lower
values of $x$~\cite{Aid:1995rk}.  
The measurements of the structure function of the proton provided
the most extensive mapping of the constituents of the protons, seen with a probe 
of varying spatial resolution.  The strong rise in the structure functions at low $x$
was a striking feature when it was discovered, but it was soon realized that perturbative
QCD could predict the radiation of gluons off of quarks, and this evolution with
increasing beam energy was able to predict the $\sqrt{s}$, $Q^2$ and $x$ dependence
of the observed nucleon structure.

The discovery of partons in the nucleon led to the prediction that hard scattering could
be possible in the collisions of two hadrons.  The signature of this process would be 
the scattering of the two partons to large angles relative to the incoming beams. 
Confinement effects would allow only the overall flow of energy to reflect the outgoing
quarks or gluons, which would necessarily fragment each into a shower of hadrons called
a ``jet''.  Jets were discovered in the annihilation of electrons and positrons to
hadrons in the mid-1970's, where it was observed even in reactions with a low center
of mass energy that the emission of outgoing hadrons was noticeably anisotropic event-to-event~\cite{Hanson:1975fe}.
However, skepticism persisted throughout the 1970's that a jet could be observed amidst the large
uncorrelated background of hadrons.
In a typical pattern, the introduction of hadron colliders in the early 1980's provided a
huge increase in the center of mass energy available for a hard scattering process,
allowing the observation of events with a clear ``di-jet'' structure, whose rates and
opening angle distributions could be predicted using perturbative QCD~\cite{Banner:1982kt}.

The connection to the structure functions measured in deep inelastic experiments to 
the production of jetes in hadronic collisions was made rigorous by the proof of several
perturbative QCD factorization theorems~\cite{Collins:1989gx}.
These theorems were important to demonstrate that processes at a hard scale had no
substantial interference terms with diagrams involving soft scales, which were 
parameterized as ``parton distribution functions'' (PDFs) and postulated to be
universal between different hadrons.
%At leading twist, high $\pT$ observables can be written as
%\begin{equation}
%E_h {d{\sigma}_{AB\rightarrow h(p')}\over d^3p'}
%=
%\sum_{ijk}\int dx' f_{j/B}(x')
%          \int dx\, f_{i/A}(x)
%          \int dz\, D_{h/k}(z)\,
%          E_h {d\hat{\sigma}_{ij\rightarrow k}\over d^3p'} 
%          (xp_A,x'p_B,\frac{p'}{z}) ,
%\label{twist2conv}
%\end{equation}
%The PDFs, indicated here as functions of $x$ and $Q^2$ for different parton constitutents ($q,g$)
%are extracted using fits to experimental data
%, under the assumption that they are universal
%for each parton.  The function $D_{h/k}$ is a fragmentation function, which are similarly based
%on experimental data, primarily from $e^+ e^-$ colliders.
In general, the description of experimental data using next-to-leading-order (NLO) perturbative calculations
along with the most recent fits to the experimental data to extract PDFs, has been a powerful validation
of the technology of pQCD. % -- and led to the 2004 Nobel Prize in Physics.

The success of the QCD factorization approach to understanding cross sections at high $\pT$
is the foundation for using so-called ``hard probes'' -- high $\pT$ hadrons, jets, heavy flavor, electroweak bosons --
to probe the hot, dense matter that is believed to be created in the collisions of nuclei at 
ultrarelativistic energies.
Violations of factorization, relative to proton-proton expectations, are expected to arise in nuclear collisions
for several reasons. 
The modification of the initial PDFs in the nuclear environment remains a topic of intense interest, since
the amount of data is small, and the physics of the modification is ultimately non-perturbative 
in origin~\cite{Eskola:2009yy}.
This is accessible typically by the measurement of colorless ``penetrating'' probes, such as photons, or electroweak
bosons.
Jet rates, as well as the rates of their fragmentation products, are expected to be modified by energy loss
of high energy quarks and gluons in the hot, dense matter by means of radiative or 
collisional energy loss~\cite{Wang:1991hta}.
The study of heavy flavor is expected to provide insight into whether the mass of the quark modifies its
interaction with the medium~\cite{Dokshitzer:2001zm}.  
Finally, the study of the suppression of charm and bottom quarkonia may provide  
insight into the initial temperature of the system by the differing binding energies of the different
available states~\cite{Matsui:1986dk}.
This review hopes to provide a starting point to understanding the increasing number of hard scattering results in
heavy ion collisions.
Results are shown from three of the large LHC experiments -- ALICE~\cite{Aamodt:2008zz}, ATLAS~\cite{Aad:2008zzm} 
and CMS~\cite{Chatrchyan:2008aa} -- who have each sampled the
approximately 150 inverse microbarns of data (corresponding to roughly a billion inelastic events) 
delivered to each interaction region.
